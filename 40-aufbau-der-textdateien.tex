\section{Aufbau der Textdateien}
\label{aufbau_der_textdateien}

Die einzelnen Textdateien werden mit zweistelligen Nummern, 
die sich am Anfang des Dateinamens befinden,
durchnummeriert.
Die Stellen 00 - 09 und 90 - 99 dienen dabei als Steuerdateien 
und können nicht verwendet werden.
Die erste Nummer die verwendet werden kann ist somit die 10.
Der \verb!cat *.tex! Befehl, 
dem man in der \verb!makefile! Datei findet, 
stellt dabei sicher dass die Dateien in der richtigen 
Reihenfolge zusammengebaut werden.



\subsection{Beispiel dieses Dokuments}

Dieses Dokument besteht aus folgenden Dateien erstellt.

\begin{itemize}
	\item 00-start.tex
	\item 08-titel.tex
	\item 09-abstract.tex
	\item 10-beschreibung.tex
	\item 20-voraussetzung.tex
	\item 30-installation.tex
	\item 40-aufbau-der-textdateien.tex
	\item 50-anmerkung.tex
	\item 99-ende.tex
	\item makefile
	\item pdfname
\end{itemize}


\subsection{Ein- Ausschalten von Textteilen}
\label{sec:ein_ausschalten_von_textteilen}

Will man Textteile ausschalten, zum Beispiel das Anzeigen
des Inhaltsverzeichnisses, so muss man die entsprechende Datei
nur umbenannt werden.
Durch die, unter Linux, automatische Dateinamen Vervollständigung%
\footnote{Diese erreicht man durch drücken der \texttt{TAB} Taste}
bietet sich das ändern des Datei Suffixes an.

\textbf{Beispiel:} Will man die Anzeige des Inhaltsverzeichnisses ausschalten,
so benennt man die Datei 
\texttt{07-tableof\-contents\-.tex} in \texttt{07-tableof\-contents\-.texx}
um.

\begin{verbatim}
mv 07-tableofcontents.tex 07-tableofcontents.texx
\end{verbatim}

Nach einem \verb!make! wird das Inhaltsverzeichnis nicht mehr angezeigt.



