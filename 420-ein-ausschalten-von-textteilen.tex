\subsection{Ein- Ausschalten von Textteilen}
\label{sec:ein_ausschalten_von_textteilen}

Will man Textteile ausschalten, zum Beispiel das Anzeigen
des Inhaltsverzeichnisses, so muss man die entsprechende Datei
nur umbenannt werden.
Durch die, unter Linux, automatische Dateinamen Vervollständigung%
\footnote{Diese erreicht man durch drücken der \texttt{TAB} Taste}
bietet sich das ändern des Datei Suffixes an.

\textbf{Beispiel:} Will man die Anzeige des Inhaltsverzeichnisses ausschalten,
so benennt man die Datei 
\texttt{07-tableof\-contents\-.tex} in \texttt{07-tableof\-contents\-.texx}
um.

\begin{verbatim}
mv 07-tableofcontents.tex 07-tableofcontents.texx
\end{verbatim}

Nach einem \verb!make! wird das Inhaltsverzeichnis nicht mehr angezeigt.



