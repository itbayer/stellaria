\section{Installation}
\label{sec:installation}

Um diese Zusammenstellung zu installieren muss das Paket
\href{http://www.it-bayer.de/download/stellaria.tar.gz}{stellaria.tar.gz} 
herunter geladen und entpackt werden.

\begin{verbatim}
wget http://www.it-bayer.de/download/stellaria.tar.gz
tar -xzvf stellaria.tar.gz
\end{verbatim}

Ein Umbenennen des Verzeichnisses ist sicher sinnvoll.

\begin{verbatim}
mv stellaria meine_pdf
\end{verbatim}

\textbf{Download Link} 

\url{http://www.it-bayer.de/download/stellaria.tar.gz}

\subsection{PDF Dateiname}
\label{ssec:pdf_dateiname}

Der PDF Dateiname wird aus der Datei \verb!pdfname!
gelesen. Diese Datei kann man mit Befehl%
\footnote{%
In dem Beispiel wird der PDF Name \texttt{stellaria.pdf} gesetzt.
}

\begin{verbatim}
echo "stellaria" > pdfname
\end{verbatim}

erstellen.






% ------------------------------------------------------
\subsection{Grund Dateien}
\label{ssec:grund-dateien}

\todo

Das Verzeichnis besteht aus folgenden Grunddaten.

\begin{itemize}
	\item 00-priamble.tex
	\item \emph{\dots 02-frei}
	\item 03-titel.tex
	\item 04-makeindex.tex
	\item 05-begin\_document.tex
	\item 06-maketitle.tex
	\item 07-tableofcontents.tex
	\item \emph{\dots 08-frei}
	\item 09-abstract.tex
	\item \emph{\dots Textteile 10 - 80}
	\item 99-ende.tex
	\item makefile
\end{itemize}





