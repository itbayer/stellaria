\section{Anmerkung}
\label{sec:anmerkung}


% ------------------------------------------------------
\subsection{make Befehle}
\label{ssec:make_befehle}

\todo

Dei Steuerung übernimmt \verb!make!, folgende Befehle stehen
dabei zur Verfügung.

\begin{itemize}
	\item \verb!make!, siehe \verb!make run!
	\item \verb!make run!      \hspace{1em} Erstellt das PDF Dokument
	\item \verb!make show&!    \hspace{1em} Zeigt das PDF
		Dokument\footnote{Bei diesem Befehl muss unter (g)vim das \& Zeichen
	gesetzt werden da sonst \texttt{make} den Editor nicht frei gibt.}
	\item \verb!make clean!    \hspace{1em} Löscht alle Arbeitsdateien. Die
		eigenen \texttt{.tex} Dateien sind natürlich nicht betroffen.
	\item \verb!make cleanall! \hspace{1em} Löscht alle Arbeitsdateien incl.
		dem PDF Dokument
\end{itemize}




\subsection{PDF Betrachter Zathura}
\label{ssec:zathura}

Dieser PDF Betrachter wurde gewählt da man diesen,
genau so wie (g)vim,
mit der Tastatur steuern kann.
Die Tasten um im Dokument zu navigieren sind dabei die gleichen.
